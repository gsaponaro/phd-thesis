%!TEX encoding = UTF-8

\begin{otherlanguage*}{portuguese}
Os robôs pessoais e de serviço podem beneficiar a sociedade em atividades e desafios diferentes, graças às suas capacidades mecânicas e de tomadas de decisão cada vez mais avançadas.
No entanto, para isso acontecer, uma das condições essenciais é a existência de interfaces coerentes, naturais e intuitivas para que os humanos possam usar estas máquinas inteligentes de uma forma frutuosa e eficaz.
No geral, o objetivo de alcançar interfaces intuitivas para a interação homem--robô (\emph{\hri}) não foi ainda alcançado, em parte pelo facto de que os robôs não estão ainda difusos nos espaços públicos, em parte devido às dificuldades técnicas na interpretação das intenções humanas.

Construir robôs de serviço, que percebam o que os rodeia, implica que os mesmos possuam capacidades que lhes permitam funcionar em ambientes não estruturados e em condições imprevisíveis, ao contrário dos robôs industriais que operam em ambientes altamente estruturados, controlados e repetíveis.
Para enfrentar estes desafios, nesta tese desenvolvemos modelos computacionais baseados em descobertas da psicologia do desenvolvimento~(potencialidades de objetos, \emph{object affordances}) e da neurociência~(sistema de neurónios espelho).
Os modelos propostos resultam da consideração que os objetos transportam informação sobre ações e interações.

Propomos uma biblioteca de \emph{software} modular para aprendizagem de potencialidades visuais em robôs, baseada na exploração autónoma do mundo, recolha de dados sensório-motores e técnicas estatísticas.
Juntamos potencialidades com comunicação (gestos e linguagem), para interpretar e descrever ações humanas em cenários de manipulação, reutilizando a experiência prévia do robô.
Mostramos um modelo que lida com objetos múltiplos, permitindo o uso de ferramentas, e também a ligação de potencialidades de mãos~(i.e., possibilidade de ações usando as mãos) para potencialidades de ferramentas~(i.e., possibilidade de ações usando ferramentas).
Explicamos como as potencialidades podem ser usadas para o planeamento de ações complexas de manipulação, sob ruído e incerteza.
Em dois apêndices,
descrevemos um modelo robótico para reconhecer gestos humanos em contextos de manipulação,
e relatamos um estudo de como as pessoas percebem gestos robóticos quando a informação facial dos mesmos é desligada.

\bigskip

\textbf{Palavras-chave}: \myPortugueseThesisKeywords
\end{otherlanguage*}
